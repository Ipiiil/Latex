\documentclass[article,12pt, fleqn]{article}

\usepackage{Headers/header}

\title{дизайн документ}
\author{--- }
\date{Ноябрь - декабрь 2024}
\begin{document}
\maketitle
\begin{center}
    \tableofcontents % Создание оглавления
\end{center}

\newpage
\section{Введение}

\section{Концепция}

\subsection{Введение}

\subsection{Жанр и аудитория}

\subsection{Основные особенности игры}

\subsection{Описание игры}

\subsection{Предпосылки создания}

\subsection{Платформа}

\section{Функциональная спецификация}

\subsection{Принципы игры}

\subsubsection{Суть игрового процесса}

\subsubsection{Ход игры и сюжет}
\subsection{Физическая модель}
\subsection{Персонаж игрока}
\subsection{Элементы игры}
\subsection{Искуственный интеллект}
\subsection{Многопользовательский режим}
\subsection{Интерфейс пользователя}
\subsubsection{Блок-схема}
\subsubsection{Функциональное описание и управление}
\subsubsection{Объекты интерфеса пользователя}
\subsection{Графика и Видео}
\subsubsection{Общее описание}
\subsubsection{Двумерная графика и анимация}
\subsubsection{Трехмерная графика и описание}
\subsubsection{Анимационные вставки}
\subsection{Звуки и музыка}
\subsubsection{Общее описание}
\subsubsection{Звук и звуковые эффекты}
\subsubsection{Музыка}
\subsection{Описание уровней}
\subsubsection{Общее описание дизайна уровней}
\subsubsection{Диаграмма взаимного расположения уровней}
\subsubsection{График введения новых объектов}
\section{Контакты}
\end{document}
