\documentclass[article,12pt, fleqn]{article}

\usepackage{Headers/header}

\title{дизайн документ}
\author{--- }
\date{Ноябрь - декабрь 2024}
\begin{document}
\maketitle
\begin{center}
    \tableofcontents % Создание оглавления
\end{center}

\newpage
\section{Введение}
\begin{itemize}
    \item Данный документ представляет собой концепцию игры «Super Marmel», описывающую ее\ \ основную\ \ идею,\ \ механику,\ \ жанр,\ \ целевую \ \ аудиторию\ \ и \ \ ключевые \ \ особенности.\\ Документ структурирован по разделам, позволяющим получить полное представление о проекте. 
    \item Ссылки на используемые материалы: 
    \item История изменений документа:
    \item Список авторов: 
    \begin{itemize}
        \item Терехина Полина Сергеевна
        \item Медведева Ульяна Андреевна
        \item Чеботова Татьяна Сергеевна
        \item Игутова Мария Дмитриевна
    \end{itemize}
    \item В документе используются стандартные термины и сокращения, принятые в игровой индустрии. Названия игры, персонажей, мест выделены кавычками.
\end{itemsize}
\section{Концепция}
"Super Marmel" — это красочный и увлекательный платформер для детей 6+, где игроки управляют одной из четырех девочек-супергероинь, чтобы спасти город "Свит" от злодеев "Аква". Игра предлагает нелинейный геймплей, где игрок самостоятельно выбирает путь и решает головоломки, используя специальные самокаты с яркими фонарями, чтобы ослепить преступников и восстановить мир в городе. 

\subsection{Введение}

\subsection{Жанр и аудитория}

\subsection{Основные особенности игры}
\begin{itemize}
    \item  Платформер 2D с 5 постепенно усложняющимися уровнями.
    \item Персонаж с базовыми действиями: бег и прыжок.
    \item 3 типа врагов с различными поведением: статичные, бегающие и летающие.
    \item Смерть при столкновении с врагом.
    \item Победа над врагами путем прыжка на них.
    \item Меню с опциями: играть, продолжить, выйти.
    \item Кнопка паузы и клавиша escape для остановки игры и выбора дальнейших действий.
    \item Переход на следующий уровень при победе над всеми врагами и достижении мармеладного мишки.
\end{itemize}

\subsection{Описание игры}
Прославьтесь в эпическом путешествии, где вы берете на себя роль бесстрашной девушки-героя. В этом захватывающем 2D-платформере вас ждет захватывающее испытание с пятью усложняющимися уровнями.\par
Управляйте своей героиней с помощью интуитивно понятных элементов управления бегом и прыжками. Внимательно изучайте каждый уровень, ловко перепрыгивая через препятствия и избегая опасных врагов. \par
Мобилизуйте уникальный талант своей героини в битве. Прыгая на врагов, вы можете победить их и очистить себе путь. Вы должны быть достаточно быстры, чтобы избежать столкновений, так как они могут убить вашу героиню.\par
Пробивайтесь через разнообразные и все более сложные уровни, каждый из которых предлагает уникальные испытания. \par
Присоединяйтесь к отважной девочке-герою в этом захватывающем 2D-платформере, где ловкость, смекалка и отвага станут ключом к победе. Испытайте себя и победите все препятствия, которые встанут на вашем пути!

\subsection{Предпосылки создания}

\subsection{Платформа}

\section{Функциональная спецификация}

\subsection{Принципы игры}

\subsubsection{Суть игрового процесса}

\subsubsection{Ход игры и сюжет}
\subsection{Физическая модель}
\subsection{Персонаж игрока}
\subsection{Элементы игры}
\subsection{Искуственный интеллект}
\subsection{Многопользовательский режим}
\subsection{Интерфейс пользователя}
\subsubsection{Блок-схема}
\subsubsection{Функциональное описание и управление}
\subsubsection{Объекты интерфеса пользователя}
\subsection{Графика и Видео}
\subsubsection{Общее описание}
\subsubsection{Двумерная графика и анимация}
\subsubsection{Трехмерная графика и описание}
\subsubsection{Анимационные вставки}
\subsection{Звуки и музыка}
\subsubsection{Общее описание}
\subsubsection{Звук и звуковые эффекты}
\subsubsection{Музыка}
\subsection{Описание уровней}
\subsubsection{Общее описание дизайна уровней}
\subsubsection{Диаграмма взаимного расположения уровней}
\subsubsection{График введения новых объектов}
\section{Контакты}
\end{document}
